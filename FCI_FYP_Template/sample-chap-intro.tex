\chapter{Introduction, Background Story, Motivations}
\section{Basic Introduction}
In the Introduction section, you should describe the problem investigated. Try to summarize relevant research to provide context, key terms, and concept so the reader can understand the whole final year project. Go and read journal or conference papers and review relevant past research to 
provide rational or justification for your work. Define clearly your final project objectives and briefly describe your research – design, 
research, hypothesis, etc. 

\begin{figure}[hbt!]\centering
\includegraphics[width=.3\textwidth]{bee}
\caption{Example of a first figure. Fig. 1}
\end{figure}

\subsection{Related Works}
There are studies on factors blah blah \cite{audibert:2004} and\endnote{This is a footnote, or rather an endnote. Note that footnotes/endnotes are not encouraged in scientific and engineering disciplines.} they are really amazing\endnote{don't you agree?}\cite{budanitsky:hirst:2006}.

\begin{figure}[hbt!]\centering
\includegraphics[width=.3\textwidth]{robot}
\caption{Example of a second figure. Figure 2}
\end{figure}

\subsubsection{This is an example of sub sub section}
It works!\index{test} Let's talk about \acp{LI} and \acp{POS} in \ac{NLP}.\index{lexical item}\index{part-of-speech} I mention again \acp{LI}. We will also talk about \glsplural{lexicon}.

\section{Another new section}
\lipsum[5-6]

\begin{table}[hbt!]
\caption{This is an example of a table}
\centering
\begin{tabular}{ l c r }
\hline
Column 1 & Column 2\\ \hline
Fine! & Just great. & See ya!\\
Fine! & Just great. & See ya!\\
\hline
\end{tabular}
\end{table}